\documentclass{letter}
\usepackage[margin=1in]{geometry}
\usepackage{hyperref}

\signature{Qiaohui Lin \\ PhD student \\Department of Statistics and Data Sciences\\ University of Texas at Austin\\ qiaohui.lin@utexas.edu}

\address{Qiaohui Lin \\ Department of Statistics and Data Sciences \\ University of Texas at Austin \\2317 Speedway D9800 \\ Austin, TX 78712-1823 }
\longindentation=0pt
\begin{document}

\begin{letter}{E.H. Shortliffe, M.D \\ Editor-in-Chief \\ Journal of Applied Statistics}
\opening{Dear Editor-in-Chief:}

My name is Qiaohui Lin, and I am writing on behalf of my collaborators, Brenda Betancourt, Benjamin Goldstein and Rebecca Steorts to submit our manuscript entitled ``Prediction of Appointment No-shows using Electronic Health Records'' for consideration for publication in the Journal of Applied Statistics.
Appointment no-shows are cause of substantial loss in resources and revenue for health care systems. Intervention strategies to reduce no-show rates can be more effective if targeted
to the subpopulations of patients with higher risk of not showing to their appointments. In this manuscript,
we use electronic health records (EHR) from a large medical center to predict no-show patients based on demographic and health care features. We explore penalized logistic regression models within a Bayesian framework to predict and identify the most relevant risk factors of no-show patients at a provider level. Machine learning approaches such as the ones presented in this paper can be implemented and adapted to hospital scheduling systems in order to inform preventive measures.

We believe that this manuscript is appropriate for publication by Journal of Applied Statistics because it is motivated by a relevant health care problem and we explore Bayesian approaches that  have not been used in this particular context to the best of our knowledge.  

Thank you for your time and consideration, and do not hesitate to contact me if you have any questions regarding our paper. 

\closing{Sincerely,}

\end{letter}
\end{document}